\documentclass[12pt,a4paper,titlepage,oneside]{scrartcl}
\newcommand{\lang}{en}
\usepackage{pp}

\usepackage{multicol}

\newcommand{\team}{2}
\newcommand{\dokumenttyp}{Report}
% Date
\newcommand{\datum}{\today}

\newcommand{\lvaname}{\ifthenelse{\equal{\lang}{de}}{Einführung paralleles
Rechnen}{Parallel Computing}} 
\newcommand{\lvanr}{184.710}
\newcommand{\semester}{WS 2016}

% Student data in Lab0 or 1. student of team in Lab1
\newcommand{\studentAName}{Ferdinand Baarlink}
\renewcommand{\studentAMatrnr}{1635879}

% 2. student of team in Lab1, for Lab0 or if your team has less students, remove these 2 lines
\newcommand{\studentBName}{Roland Wallner}
\renewcommand{\studentBMatrnr}{1427019}



%%%%%%%%%%%%%%%%%%%%%%%%%%%%%%%%%%%%%%%%%%%%%%%%%%%%%%%%%%%%%%%%%%%%%%
%
% DO NOT CHANGE THE FOLLOWING PART
%
%%%%%%%%%%%%%%%%%%%%%%%%%%%%%%%%%%%%%%%%%%%%%%%%%%%%%%%%%%%%%%%%%%%%%%

\newcommand{\colormode}{color}


\begin{document}


%%%%%%%% TITLE PAGE %%%%%%%%%%%%%%%%%%

\begin{center}
\vspace{2.5cm}
{\LARGE\textbf \dokumenttyp\\}
\vspace{0.5cm}
{\LARGE\textbf \lvaname\\}
\vspace{0.5cm}
{\LARGE \lvanr\ -- \semester\\}
\vspace{1.0cm}
{\LARGE \datum}
\vspace{1.5cm}

{\LARGE \ifthenelse{\equal{\lang}{de}}{Gruppe}{Group} \team}
\vspace{1.5cm}

\teamnMitglieder


\end{center}

\vspace{1.5cm}

\newpage
%%%%%%%% TITLE PAGE ENDS HERE %%%%%%%%

%\setcounter{section}{0}
%\setcounter{tocdepth}{2}
\tableofcontents

%%%%%%%%%%%%%%%%%%%%%%%%%%%%%%%%%%%%%%%%%%%%%%%%%%%%%%%%%%%%%%%%%%%%%%
%
% CONTENT OF DOCUMENT STARTS HERE
%
%%%%%%%%%%%%%%%%%%%%%%%%%%%%%%%%%%%%%%%%%%%%%%%%%%%%%%%%%%%%%%%%%%%%%%

\section{Introduction}
Lorem ipsum dolor sit amet, consetetur sadipscing elitr, sed diam nonumy eirmod tempor invidunt ut labore et dolore magna aliquyam erat, sed diam voluptua. At vero eos et accusam et justo duo dolores et ea rebum. Stet clita kasd gubergren, no sea takimata sanctus est Lorem ipsum dolor sit amet. Lorem ipsum dolor sit amet, consetetur sadipscing elitr, sed diam nonumy eirmod tempor invidunt ut labore et dolore magna aliquyam erat, sed diam voluptua. At vero eos et accusam et justo duo dolores et ea rebum. Stet clita kasd gubergren, no sea takimata sanctus est Lorem ipsum dolor sit amet.

\section{Solution}

\subsection{Co-ranking}

\subsection{Implementation}

\subsubsection{OpenMP}

\subsubsection{Cilk}

\subsubsection{MPI}

\section{Testing}

\section{Benchmark setup}
For our tests we measured the time that the operation needed with the appropiate methods. So we used \textit{omp\_get\_wtime} for OpenMP, \textit{clock\_gettime} for Cilk and \textit{MPI\_Wtime} for MPI.

First we decided which problemsizes would be part of the benchmark. We agreed on the following sizes (m, n) where both input arrays will have this size:

\begin{multicols}{3}
\begin{itemize}
\item 50
\item 100
\item 1000
\item 10000
\item 100000
\item 500000
\item 1000000
\item 5000000
\item 10000000
\end{itemize}
\end{multicols}

Additionally we did tests where the first array has a different size than the second one.
\begin{center}
\begin{tabular}{l | r | r | r | r | r}
\hline
 Type & 1 & 2 & 3 & 4 \\ \hline
 m & 150000 & 50000 & 1500000 & 500000 \\ \hline
 n & 50000 & 150000 & 500000 & 1500000 \\ \hline
\end{tabular}
\end{center}

Next to the size of the arrays we needed to specify the layout of the values inside the arrays.
There is four different cases we benchmarked:

\begin{description}
\item[Interleaved] The numbers alternate between the two arrays. (e.g: array1 = \{ 1, 3, 5, 7, 9, ...\}, array2 = \{ 2, 4, 6, 8, 10, ...\})
\item[Smaller First] All numbers in the first array are smaller than the ones in the second one. (e.g: array1 = \{ 1, 2, 3, 4, 5, ...\}, array2 = \{ 10, 11, 12, 13, ...\})
\item[Smaller Last] All numbers in the first array are bigger than the ones in the second one.  (e.g: array1 = \{ 10, 11, 12, 13, ...\}, array2 = \{ 1, 2, 3, 4, 5, ...\})
\item[Random] The numbers in both arrays are still sorted, but cannot know which number is in which array. (e.g: array1 = \{ 1, 3, 4, 7, ... \}, array2 = \{2, 5, 6, 9, ... \})
\end{description}

After specifying which problemsizes and the array layouts we want to test. We specified the rest of the testing parameters. We decieded to do \textbf{35 repetitions} for each test and only use the \textbf{median time} value for further calculations.

\section{Results}


\subsection{Expected Results}

\subsection{Real Results}


\section{Summary}


\section{Beispiele}
\subsection{Source Code formatieren}
Es folgen einige Beispiele wie Sourcecode in diesem Dokument formatiert und referenziert werden kann
(\hyperref[code:beispiel1]{siehe Listing~\ref*{code:beispiel1} auf Seite~\pageref*{code:beispiel1}} und \hyperref[code:beispiel2]{siehe Listing~\ref*{code:beispiel2} auf Seite~\pageref*{code:beispiel2}}).

Ebenso können kurzer Code oder kurze Befehle direkt in der Zeile in einem \lstinline{lstinline Block} mit typengleicher Schrift formatiert werden.

\lstinputlisting[caption=Example C/C++ file,label=code:beispiel1,style=c]{corank.c}

\begin{lstlisting}[caption=Example bash script,label=code:beispiel2,style=simple]
#!/bin/bash
echo "Bash version ${BASH_VERSION}..."
for i in {0..10..2}
  do
     echo "Welcome $i times"
 done

echo "some very very very very very very very very very very very very very very very very very very very very long string"

exit 0;
\end{lstlisting}

\subsection{Bilder}

Es folgen einige Beispiele wie Bilder in diesem Dokument eingefuegt werden koennen
%(\hyperref[fig:cilk01]{siehe Abbildung~\ref*{fig:cilk01} auf
%Seite~\pageref*{fig:cilk01}}).

%\begin{figure}[h!]
  %\centering
  %\fbox{
    %\includegraphics[width=0.8\textwidth]{./imgs/graph1.png} %width in prozent.
  %}
  %\caption{Cilk: n = 1 000 000}
  %\label{fig:cilk01}
%\end{figure}


%%%%%%%%%%%%%%%%%%%%%%%%%%%%%%%%%%%%%%%%%%%%%%%%%%%%%%%%%%%%%%%%%%%%%%
%
% DO NOT CHANGE THE FOLLOWING PART
%
%%%%%%%%%%%%%%%%%%%%%%%%%%%%%%%%%%%%%%%%%%%%%%%%%%%%%%%%%%%%%%%%%%%%%%

\end{document}


